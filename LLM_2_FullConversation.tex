This conversation was held with Anthropic's Claude 3.7 Sonnet in May of 2025. The prompt used to generate the conversation can be found in Appendix \ref{appendix:LLMMetaPrompt}. Other documents provided to Claude for this conversation include:
\begin{itemize}
    \item CPSTRIDE specification
    \item STRIDE specification \cite{kohnfelder_threats_1999,microsoft_uncover_2006}
    \item AM System CPFD (Figure \ref{AMCPFD})
    \item AM System DFD (Figure \ref{AMDFD})
\end{itemize}

\subsection{User}
Hello. Please identify all elements in the AMCPFD for which there are no corresponding elements in the AMDFD.

\subsection{Claude 3.7 Sonnet LLM}
Elements in CPFD with No Corresponding Elements in DFD
1. Physical and Cyber-Physical Devices (CPD)

CPD3: AM Printer - The physical 3D printer hardware is not represented in the DFD
CPD4: AM Debinder - The physical debinder hardware is not represented in the DFD
CPD5: AM Furnace - The physical sintering furnace hardware is not represented in the DFD

2. Physical and Cyber-Physical Links (CPL)

CPL1-CPL7 - The physical connection paths and media (ethernet cables, RF transmissions) that enable flows between elements are not represented in the DFD

3. Physical and Cyber-Physical Stores (CPS)

CPS1: AM Part Material Cartridge - The physical feedstock/material is not represented in the DFD
CPS2: Printed Part - The physical "green" part after printing is not represented in the DFD
CPS3: Debound Part - The physical part after debinding is not represented in the DFD
CPS4: Sintered Part - The physical part after sintering is not represented in the DFD
CPS5: Finished Part - The physical finished part after quality control is not represented in the DFD

4. Physical Flows (PF)

PF1 - The physical inspection flow from human operator to part during quality control is not represented in the DFD

5. Cyber-Physical Flows (CPF)

CPF3-CPF12 - Flows involving physical material movement or physical part handling are not represented in the DFD
CPF13 - Material flow from supply chain provider is not represented in the DFD
CPF14 - Finished part flow to downstream manufacturer is not represented in the DFD
CPF15 - Physical operator control interactions with machines are not represented in the DFD

6. Cyber-Physical Interactors (CPI)

CPI3: Material Supply Chain Provider - The external material supplier is not represented in the DFD
CPI4: Critical Downstream Manufacturing - The recipient of the manufactured parts is not represented in the DFD

7. Cyber-Physical Processes (CPP)

CPP1-CPP4 - The physical aspects of the manufacturing processes (printing, debinding, sintering, quality control) are not represented in the DFD, which only depicts their digital control aspects

This comparison highlights that the DFD focuses exclusively on digital data flows and information processing, while omitting the physical elements and interactions that are critical to additive manufacturing. The CPFD provides a more comprehensive view of the entire cyber-physical system, capturing both the digital and physical dimensions of the AM process.

\subsection{User}
Create a CPSTRIDE threat identification table for the AMCPFD. Each table row represents an element from the AMCPFD, and each table column represents a CPSTRIDE threat category (Spoofing, Tampering, Repudiation, Interception, Denial of Service, Elevation of Privilege). Use only the elements identified in the previous step (present only in AMCPFD, not AMDFD). The goal is to illustrate the threats considered by the CPSTRIDE process that STRIDE would not consider. Thus, for the Cyber-Physical Processes identified, focus on the physical and cyber-physical threats rather than purely cyber threats that STRIDE would also identify. For each cell of the table, consider the threat category and element in the following contexts: * Is the element the subject (performing the threat category action)? * Is the element the object (the target/victim of the threat category)? * Is the element an instrument (used as a means to threaten another target)? Identify realistic threats only; it is not necessary that the table be full, only that all realistic threats be considered. The table will be quite extensive with many elements. Start from the beginning and work through systematically. If needed, complete a small number of rows at a time, and await prompting to continue building the table. If there are groups of elements with similar or identical threats, you may group them together to reduce redundancy in the table. Format as a readable table in your response.

\subsection{Claude 3.7 Sonnet LLM}

CPSTRIDE Threat Identification Table for Physical and Cyber-Physical Elements

I'll create a threat identification table focusing on the elements that are present only in the CPFD and not in the DFD. This will illustrate the additional threats that CPSTRIDE can identify but STRIDE would miss. I'll group similar elements together where appropriate and consider each element's potential role as subject, object, or instrument of threats.

(creates tables)

This represents the first portion of the table. Would you like me to continue with the remaining elements (Physical Flows, Cyber-Physical Flows, Cyber-Physical Interactors, and Cyber-Physical Processes)?

\subsection{User}
Yes, continue.

\subsection{Claude 3.7 Sonnet LLM}
CPSTRIDE Threat Identification Table (Continued)
(continues creating tables)
This completes the CPSTRIDE threat identification table for the physical and cyber-physical elements present only in the CPFD. As shown, CPSTRIDE identifies numerous threats that would not be captured in a traditional STRIDE analysis focused solely on the digital domain.
The table illustrates how physical aspects of the manufacturing system introduce unique security considerations across all threat categories:

Physical components can be spoofed, counterfeited, or misrepresented
Physical processes can be tampered with through environmental manipulation
Physical interactions can lack accountability mechanisms
Physical phenomena can leak information through side channels
Physical components can be damaged or obstructed in ways that deny service
Physical access can enable privilege escalation through bypass of logical controls

These physical and cyber-physical threats demonstrate the value of expanding beyond traditional STRIDE to fully capture the security implications of cyber-physical systems like additive manufacturing.